\chapter{Технологическая часть}

В данном разделе представленны средства разработки программного обеспечения, детали реализации и тестирование функций.

\section{Средства реализации}

В качестве языка программирования, на котором будет реализовано программное обеспечение, выбран язык программирования C++. Выбор языка обусловлен тем, что на нем написана библиотека Qt, следовательно наиболее полный функционал доступен для этого языка. 

Библиотека Qt была выбрана потому что я имею опыт работы с ней, а так же она имеет достаточно обширную возможность работы с изображениями (QImage).

Для работы с DNG-файлами была использована библиотека TinyDNGLoader, она предоставляет доступ ко всем необходимым полям, при этом не является полноценной библиотекой для обработки raw-изображений как, например, libraw.

Для обеспечения качества кода был использован инструмент clang-tidy, позволяющий во время процесса написания исходных кодов программного обеспечения контролировать наличие синтаксических и логических ошибок.

В качестве среды разработки выбран текстовый редактор Qt Creator, он позволяет удобно проектировать Qt интерфейсы, в него интегрирован gdb, а так же он отображает предупреждения clang-tidy прямо во время редактирования.

\section{Реализация алгоритмов}

В листинге \ref{lst:bilinear} представлен объект реализующий алгоритм билинейной интерполяции. В листинге \ref{lst:mhc} представлен объект реализующий алгоритм интерполяции Malvar-He-Cutler.

\section*{Вывод}

