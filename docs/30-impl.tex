\chapter{Технологическая часть}

В данном разделе представленны средства разработки программного обеспечения, детали реализации и тестирование функций.

\section{Средства реализации}

В качестве языка программирования, на котором будет реализовано программное обеспечение, выбран язык программирования C++. \cite{cpp} Выбор языка обусловлен тем, что на нем написана библиотека Qt, следовательно наиболее полный функционал доступен для этого языка. 

Библиотека Qt была выбрана потому что автор работы имеет опыт работы с ней, а так же она имеет достаточно обширную возможность работы с изображениями (QImage). \cite{qt}

Опционально используется библиотека OpenMP для параллельного выполнения алгоритмов. \cite{openmp}

Для работы с DNG-файлами была использована библиотека TinyDNGLoader, она предоставляет доступ ко всем необходимым полям, при этом не является полноценной библиотекой для обработки raw-изображений как, например, libraw. \cite{tinydng}

Для обеспечения качества кода был использован инструмент clang-tidy, позволяющий во время процесса написания исходных кодов программного обеспечения контролировать наличие синтаксических и логических ошибок. \cite{clang-tidy}

В качестве среды разработки выбран текстовый редактор Qt Creator, \cite{qtc} он позволяет удобно проектировать Qt интерфейсы, в него интегрирован gdb, \cite{gdb} а так же он отображает предупреждения clang-tidy прямо во время редактирования.

\section{Реализация алгоритмов}

В листинге \ref{lst:mhc} представлен объект реализующий алгоритм Malvar-He-Cutler. В листинге \ref{lst:shmhc} представлен объект реализующий алгоритм интерполяции Smooth hue transition + Malvar-He-Cutler.

\begin{lstinputlisting}[
	caption={Malvar-He-Cutler.},
	label={lst:mhc},
	style={cpp},
	]{../src/mhc_debayer.cpp}
\end{lstinputlisting}

\begin{lstinputlisting}[
	caption={Smooth hue transition + Malvar-He-Cutler.},
	label={lst:shmhc},
	style={cpp},
	]{../src/sh_mhc_debayer.cpp}
\end{lstinputlisting}


\section*{Вывод}

В данном разделе были рассмотрены средства, с помощью которых было реализовано ПО, а также представлены листинги кода с реализацией описанных алгоритмов.

