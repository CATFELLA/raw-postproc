\chapter{Аналитическая часть}

В данном разделе описаны необходимые для обработки данных с сенсора алгоритмы.

\section{Процесс обработки кадра.}

Raw-кадр является набором значений с матрицы, поэтому для показа без обработки не пригоден.

Типичная обработка включает в себя:
\begin{itemize}
	\item декодинг данных, например если каждому пикселю соответствуют 14 бит информации они, скорее всего, лежат последовательно и их придется декодировать;
	\item демозаика, то есть устранение фильтра Байера;
	\item преобразование цвета из пространства цвета камеры в общепринятое;
	\item изменение гаммы и прочих параметров изображения.
\end{itemize}

На изображении \ref{img:big_debayer} показан ожидаемый результат обработки.

\img{50mm}{big_debayer}{Ожидаемый результат обработки. Изображение интерполировано при помощи билинейной интерполяции, проведена коррекция цвета, увеличена яркость и контрастность изображения.} 

\section{Формат файла.}

В таблице \ref{tbl:info} указана необходимая информация о изображении для каждого из этапов.

\begin{table}[h]
	\begin{center}
		\caption{Используемая информация о изображении.}
		\label{tbl:info}
		\resizebox{16cm}{!}{
			\begin{tabular}{|c|c|}
				\hline
				\bfseries Этап обработки & \bfseries Необходимая информация \\ \hline
				Распаковка & BitsPerSample \\ \hline
				Предварительная обработка & WhiteLevel, BlackLevel \\ \hline
				Демозаика & CFALayout \\ \hline
				Коррекция цвета & ColorMatrix, CameraCalibration \\ \hline
		\end{tabular}}
	\end{center}
\end{table}

\subsection{DNG}

Digital Negative (DNG) --- формат хранения raw-изображений, основан на формате изображений TIFF. CinemaDNG является форматом хранения raw-видео и представляет собой набор DNG файлов. 

DNG хранит в себе изображение и набор метаданных, для обработки кадра используются следующие поля (TIFF tags):
\begin{itemize}
	\item BitsPerSample --- количество битов для описания каждого сэмпла (пикселя), поддерживаемые значения от 8 до 32 бит на семпл. Если BitsPerSample не равен 8 или 16 или 32, тогда биты должны быть упакованы в байты с использование стандартного порядка для TIFF FillOrder 1 (big-endian);
	\item BlackLevel --- уровень черного, все значения меньшие или равные ему считаются минимальными. Тип может быть SHORT, LONG или RATIONAL, Tag = 50714;
	\item WhiteLevel --- уровень белого, все значения больше или равные ему считаются максимальными. Тип может быть SHORT или LONG, Tag = 50717;
	\item CFALayout --- поле описывающее поддерживаемые форматы фильтра Байера. Тип --- SHORT, Tag = 50711;
	\item ColorMatrix --- матрица для преобразования из цветового пространства XYZ в цветовое пространство камеры. Тип --- SRATIONAL, Tag = 50721;
	\item CameraCalibration --- матрица для преобразования из идеального цветового пространства камеры в цветовое пространство конкретной камеры. Тип --- SRATIONAL, Tag = 50723. \cite{dngspec}
\end{itemize}

Для обработки файлов можно использовать библиотеку \guillemotleft Tiny DNG Loader\guillemotright, эта библиотека небольшая по размеру и поддерживает необходимые поля.
\section{Алгоритмы демозаики.}

Одной из главных задач обработки RAW видео является устранение эффектов фильтра Байера. Схематичное изображение фильтра Байера показано на рисунке \ref{img:bayer_num}, а на изображении \ref{img:rock_closeup_bayed} показано изображение с сенсора камеры, содержащее этот эффект.

\img{60mm}{bayer_num}{Пронумерованный фильтр Байера.} 
\img{70mm}{rock_closeup_bayed}{Изображение с сенсора камеры.}

\newpage 

\subsection{Билинейная интерполяция}
Билинейная интерполяция использует среднее значение четырех соседних пикселей соответствующего цвета, например: значения зеленого пикселя для красных или синих пикселей находятся по \ref{bill}:
\begin{equation}
	\label{bill}
	\hat{G}^{bl}(i,j) = \frac{1}{4}(G(i-1,j) + G(i+1,j) + G(i,j-1) + G(i,j+1))
\end{equation}
где $G(x,y)$ --- значение зеленого цвета в пикселе $x,y$. \cite{mhcd}

Билинейная интерполяция красного и синего канала похожи на интерполяцию зеленого, но используют пиксели лежащие по диагонали от интерполируемого.

Данный алгоритм считается одним самых быстрых и часто используется для интерполяции видео в реальном времени. \cite{bilinear}

\subsection{Malvar-He-Cutler}

Метод является улучшением билинейной интерполяции.

Улучшение достигается при помощи использования Laplacian cross-channel correction. 

Зеленый канал для красного пикселя вычисляется как \ref{gcomp}
\begin{equation}
	\label{gcomp}
	\hat{G}(i,j) = \hat{G}^{bl}(i,j) + \alpha\Delta_{R}(i,j) 
\end{equation}
где $\Delta_{R}$ --- дискретный лаплассиан красного канала по 5 точкам. По формуле \ref{rlap}.

\begin{equation}
	\label{rlap}
	\Delta_{R}(i,j) = R(i,j) - \frac{1}{4} (R(i-2,j) + R(i+2,j)+R(i,j-2)+R(i,j+2))
\end{equation}

Красный канал для зеленого пикселя вычисляется как \ref{rgcomp}

\begin{equation}
	\label{rgcomp}
	\hat{R}(i,j) = \hat{R}^{bl}(i,j) + \beta\Delta_{G}(i,j) 
\end{equation}
где $\Delta_{G}$ --- дискретный лаплассиан зеленого канала, по 9 точкам.

Красный канал для синего пикселя вычисляется как \ref{rbcomp}

\begin{equation}
	\label{rbcomp}
	\hat{R}(i,j) = \hat{R}^{bl}(i,j) + \gamma\Delta_{B}(i,j) 
\end{equation}
где $\Delta_{B}$ --- дискретный лаплассиан синего канала по 5 точкам.

Синие компоненты высчитываются так же, как и для красного.

Параметры $\alpha, \beta$ и $\gamma$ отвечают за силу корректировки, оптимальные значения расчитаны авторами алгоритма \cite{mhcd}:

\begin{equation}
	\alpha = \frac{1}{2}, \quad \beta = \frac{5}{8}, \quad \gamma = \frac{3}{4}
\end{equation}

Демозаика производится при помощи линейных фильтров, существуют 8 различных фильтров, они показаны на рисунке \ref{img:mhcp}

\newpage

\img{110mm}{mhcp}{Используемые фильтры, коэффиценты домножены на 8.} 

\subsection{Smooth hue transition}

Алгоритм работает в два прохода. Во время первого прохода восстанавливается зеленый канал, при помощи билинейной интерполяции. \cite{sht}

Затем, во время второго прохода восстанавливаются красный и синий каналы.

Синий канал для зеленого пикселя вычисляется как \ref{shbcomp} или \ref{shbcomp}:
\begin{equation}
	\label{shbcomp}
	{B}(i,j) = \frac{{G}(i,j)}{2} * (\frac{{B}(i - 1,j)}{{G}(i-1,j)} + \frac{{B}(i+1,j)}{{G}(i+1,j)})
\end{equation}
если соседи синие.

\begin{equation}
	\label{shbcomp2}
	{B}(i,j) = \frac{{G}(i,j)}{2} * (\frac{{B}(i,j-1)}{{G}(i,j-1)} + \frac{{B}(i,j+1)}{{G}(i,j+1)})
\end{equation}
если соседи красные.

Красный канал для зеленого пикселя вычисляется как \ref{shrcomp} или \ref{shrcomp}:
\begin{equation}
	\label{shrcomp2}
	{R}(i,j) = \frac{{G}(i,j)}{2} * (\frac{{R}(i,j-1)}{{G}(i,j-1)} + \frac{{R}(i,j+1)}{{G}(i,j+1)})
\end{equation}
если соседи синие.

\begin{equation}
	\label{shrcomp}
	{R}(i,j) = \frac{{G}(i,j)}{2} * (\frac{{R}(i - 1,j)}{{G}(i-1,j)} + \frac{{R}(i+1,j)}{{G}(i+1,j)})
\end{equation}
если соседи красные.

Красный канал для синего пикселя вычисляется как \ref{shrbcomp}:
\begin{equation}
	\begin{split}
	\label{shrbcomp}
	{R}(i,j) = \frac{{G}(i,j)}{4} * (\frac{{R}(i-1,j-1)}{{G}(i-1,j-1)} + \frac{{R}(i-1,j+1)}{{G}(i-1,j+1)} \\ + \frac{{R}(i+1,j-1)}{{G}(i+1,j-1)} + \frac{{R}(i+1,j+1)}{{G}(i+1,j+1)})
	\end{split}
\end{equation}

Синий канал для красного пикселя вычисляется как \ref{shbrcomp}:
\begin{equation}
	\begin{split}
		\label{shbrcomp}
		{B}(i,j) = \frac{{G}(i,j)}{4} * (\frac{{B}(i-1,j-1)}{{G}(i-1,j-1)} + \frac{{B}(i-1,j+1)}{{G}(i-1,j+1)} \\ + \frac{{B}(i+1,j-1)}{{G}(i+1,j-1)} + \frac{{B}(i+1,j+1)}{{G}(i+1,j+1)})
	\end{split}
\end{equation} 

\subsection{Smooth hue transition + Malvar-He-Cutler}
Возможным улучшением предыдущего алгоритма является использование коррекции Malvar-He-Cutler при интерполяции зеленого канала.

Второй проход остается идентичным.

\section{Цветовая модель.}
Цветовая модель --- это математическая модель описания представления цветов в виде кортежей чисел, называемых цветовыми компонентами или цветовыми координатами. Изображение с матрицы находится в цветовой модели камеры и для правильного представления картинки необходимо преобразование цветов. Например, формат DNG хранит в себе матрицу для преобразования изображения из цветовой модели камеры в CIE XYZ D50. Однако не все камеры используют свое цветовое пространство, например для камеры Canon 650D эта матрица --- единичная.

\subsection{CIE XYZ.}
В цветовой модели CIE XYZ каждый элемент кортежа примерно соответствует одной из колбочек человеческого глаза: X --- длинноволновым, Y --- средневолновым и Z --- коротковолновым. На рисунке \ref{img:cie_xyz} показана хроматическая диаграмма модели.

\img{60mm}{cie_xyz}{Хроматическая диаграмма модели CIE XYZ.} 

\subsection{ProPhoto RGB.}
Цветовая модель в которой значения кортежа означают значения основных цветов: красного, зеленого и синего. Остальные цвета получаются сочетанием базовых. Цвета такого типа называются аддитивными.

Цветовая модель ProPhoto RGB покрывает 90\% возможных цветов модели CIELAB и является рекомендованной в спецификации DNG цветовой моделью. \cite{dngspec}

 На рисунке \ref{img:prophoto} показано цветовое покрытие, по сравнению с CIE XYZ.

\img{60mm}{prophoto}{Хроматическая диаграмма модели ProPhoto RGB.} 

\section{Преобразования цветовой модели.}

Для преобразования часто используются матрицы. Пусть $CMМ$ --- матрица преобразующая XYZ D50 в цветовое пространство камеры, тогда $CM^{-1}$ будет матрицей переводящей цветовое пространство матрицы в XYZ D50. Пусть $XTP$:

\begin{equation}
	\label{XTP}
	XTP = \begin{bmatrix}
			1.3460 & -0.2556 & -0.0511 \\
			-0.5446 & 1.5082 & 0.0205 \\
			0.0 & 0.0 & 1.2123 
		\end{bmatrix}
\end{equation}

матрица преобразующая XYZ D50 в ProPhoto RGB. Тогда для преобразования изображения из цветового пространства камеры в ProPhoto RGB необходимо произвести умножение:

\begin{equation}
	\label{tosrgb}
	T_{ProPhoto RGB} = XTP * CM * T_{CC}
\end{equation}

где $T_{CC}$ --- кортеж с значениями цвета в пространстве камеры.

$T_{ProPhoto RGB}$ находится в цветовом пространстве ProPhoto RGB, но яркость все еще закодирована линейно, для правильного отображения необходимо применить гамма-коррекцию:

\begin{equation}
	\label{gamma}
	\gamma(u) = 
	\begin{cases}
		16u, & u \le 0.001953 \\
		u^{1/1.8}
	\end{cases}
\end{equation}

где $u$ --- одна из компонент цвета. \cite{ppspec}

\section{Настройка изображения.}

Значением яркости в моделях RGB считается среднее значение основных цветов:

\begin{equation}
	\label{br}
	brv = \frac{R + G + B}{3}
\end{equation}
где $R$, $G$, $B$ --- красная, зеленая и синяя компоненты пикселя соответственно. \cite{colorfaq}

Контрастность определяется как:
\begin{equation}
	\label{condef}
	C_{ip} = \frac{Li_{max} - Li_{min}}{D}
\end{equation}
где $Li_{max}$ --- максимальное, а $Li_{min}$ --- минимальное значение яркости на изображении. $D$ --- максимальное значение разности $Li_{max} - Li_{min}$. \cite{contrastdef}

Насыщенностью определяет насколько цвета различаются друг от друга, влияет на красочность изображения. Находится на промежутке от чистого цвета (100\%) до серого (0\%). \cite{satdef}

Баланс цветов, в RGB, это соотношение между основными цветами. В модели RGB у серых цвета компоненты цветов должны быть равны ($R = G = B$), то есть быть сбалансированы. В случае если они не равны изображение будет иметь оттенок. \cite{colorb}

Выбранные параметры позволят адекватно настроить световое (яркость, контрастность) и цветовое (насыщенность, баланс цветов) отношение между пикселями. 

\subsection{Яркость.}

Преобразование яркости определяется как:
\begin{equation}
	\label{brightness}
	br(a) = a + N
\end{equation}
где $a$ --- значение яркости пикселя, а $N$ --- желаемое увеличение в яркости. \cite{brk}

На рисунке \ref{img:to_bright} показан пример увеличения яркости.

\img{50mm}{to_bright}{Пример увеличения яркости.} 

\subsection{Контрастность.}

Преобразование контрастности определяется как:
\begin{equation}
	\label{contrast}
	con(a) = a * N
\end{equation}
где $a$ --- значение яркости пикселя, а $N$ --- желаемое увеличение в контрастности (чтобы, например, увеличить контрастность на 50\% необходимо умножить на 1.5). \cite{brk}

На рисунке \ref{img:to_con} показан пример увеличения контрастности.

\img{50mm}{to_con}{Пример увеличения контрастности.} 

\subsection{Насыщенность.}

Для изменения насыщенности изображения в цветовой модели RGB можно воспользоваться умножением матриц.

Пусть $F(x,y)$ --- вектор:
\begin{equation}
	\label{F}
	F(x,y) = [f_R, f_G, f_B, 1]^T
\end{equation}
где $f_R, f_G, f_B$ --- значения цвета в точке $x,y$.

Тогда $G(x,y)$ --- вектор содержащий значения цвета с иной насыщенностью:
\begin{equation}
	\label{G}
	G(x,y) = [g_R, g_G, g_B, g_w]^T
\end{equation}
где $g_R, g_G, g_B$ --- новые значения цвета в точке $x,y$, а $g_w$ не используется.

Высчитать $G(x,y)$ можно по формуле \ref{sat}:
\begin{equation}
	\label{sat}
	G(x,y) = T * F(x,y)
\end{equation}

где $T$ --- матрица преобразования:
\begin{equation}
	\label{tsat}
	T_{sat}(s) = \begin{bmatrix}
					\alpha + s & \beta & \gamma & 0\\
				    \alpha & \beta + s & \gamma & 0 \\ 
					\alpha & \beta & \gamma + s & 0 \\ 
					0 & 0 & 0 & 1 \\ 
				\end{bmatrix}
\end{equation}
где $\alpha = 0.3086(1 - s)$, $\beta = 0.6094(1 - s)$ и $\gamma = 0.0820(1 - s)$.

Значения $s < 1$ приводят к уменьшению насыщенности, значения $> 1$ --- к увеличению. \cite{sat}

На рисунке \ref{img:to_sat} показан пример увеличения насыщенности.

\img{50mm}{to_sat}{Пример увеличения насыщенности.} 

\subsection{Баланс цветов.}

Для изменения баланса цветов в цветовой модели RGB можно воспользоваться умножением матриц.

Пусть $F(x,y)$ --- вектор:
\begin{equation}
	\label{F}
	F(x,y) = [f_R, f_G, f_B, 1]^T
\end{equation}
где $f_R, f_G, f_B$ --- значения цвета в точке $x,y$.

Тогда $G(x,y)$ --- вектор содержащий значения цвета с иным балансом цвета:
\begin{equation}
	\label{G}
	G(x,y) = [g_R, g_G, g_B]^T
\end{equation}
где $g_R, g_G, g_B$ --- новые значения цвета в точке $x,y$.

Высчитать $G(x,y)$ можно по формуле \ref{wb}:
\begin{equation}
	\label{wb}
	G(x,y) = T * F(x,y)
\end{equation}

где $T$ --- матрица преобразования:
\begin{equation}
	\label{twb}
	T_{sat}(s) = \begin{bmatrix}
		R_{wb} & 0 & 0 \\
		0 & G_{wb} & 0 \\ 
		0 & 0 & B_{wb} 
	\end{bmatrix}
\end{equation}
где $R_{wb}$, $G_{wb}$ и $B_{wb}$ --- коэффициенты для каждого из основных цветов.

На рисунке \ref{img:to_cb} показан пример исправления баланса цветов.

\img{50mm}{to_cb}{Пример восстановления баланса цветов.} 
\clearpage
\section*{Вывод}

В данном разделе был проведен обзор необходимых для реализации алгоритмов. В таблице \ref{tbl:info} проведена оценка сложности каждого из этапов.

\begin{table}[h]
	\begin{center}
		\caption{Используемая информация о изображении.}
		\label{tbl:info}
		\resizebox{16cm}{!}{
			\begin{tabular}{|c|c|}
				\hline
				\bfseries Этап обработки & \bfseries Оценка сложности (лучший случай --- худший случай) \\ \hline
				Распаковка & 2 * высота + $9N$ --- 2 * высота + $41N$ \\ \hline
				Предварительная обработка & 2 * высота + $14N$ \\ \hline
				Демозаика & 2 * высота + $75N$ --- 2 * высота + $50N^2$  \\ \hline
				Коррекция цвета & 2 * высота + $74N$ \\ \hline
		\end{tabular}}
	\end{center}
\end{table}

где $N$ --- количество пикселей.