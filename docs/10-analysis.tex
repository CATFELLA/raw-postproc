\chapter{Аналитическая часть}

В данном разделе описаны необходимые для обработки данных с сенсора алгоритмы.

\section{Процесс обработки кадра.}

Raw-кадр является набором значений с матрицы, поэтому для показа без обработки не пригоден.

Типичная обработка включает в себя:
\begin{itemize}
	\item декодинг данных, например если каждому пикселю соответствуют 14 бит информации они, скорее всего, лежат последовательно и их придется декодировать;
	\item демозаика, то есть устранение фильтра Байера;
	\item преобразование цвета из пространства цвета камеры в общепринятое;
	\item изменение гаммы и прочих параметров изображения.
\end{itemize}

\section{Алгоритмы демозаики.}

Одной из главных задач обработки RAW видео является устранение эффектов фильтра Байера.

\img{60mm}{bayer_num}{Пронумерованный фильтр Байера.} 

\subsection{Билинейная интерполяция}
Билинейная интерполяция использует среднее значение двух или четырех соседних пикселей соответствующего цвета, например: значения синего и красного цвета для пикселя 8 находится по формулам \ref{bill}:
\begin{equation}
	\label{bill}
	B8 = \frac{B7 + B9}{2},
	\quad	
	R8 = \frac{R3 + R13}{2}
\end{equation}

Данный алгоритм считается одним самых быстрых и часто используется для интерполяции видео в реальном времени.

\subsection{AHD там или какой-нитб}

\section{Цветовая модель.}
Цветовая модель --- это математическая модель описания представления цветов в виде кортежей чисел, называемых цветовыми компонентами или цветовыми координатами. Изображение с матрицы находится в цветовой модели камеры и для правильного представления картинки необходимо преобразование цветов. Например, формат DNG хранит в себе матрицу для преобразования изображения из цветовой модели камеры в CIE XYZ D50.

\subsection{CIE XYZ.}
В цветовой модели CIE XYZ каждый элемент кортежа соответствует одной из колбочек человеческого глаза: X --- длинноволновым, Y --- средневолновым и Z --- коротковолновым.

\subsection{sRGB.}
Цветовая модель в которой значения кортежа означают значения основных цветов: красного, зеленого и синего. Остальные цвета получаются сочетанием базовых. Цвета такого типа называются аддитивными.

\section{Преобразования цветовой модели.}

Для преобразования часто используются матрицы. Пусть $CMМ$ --- матрица преобразующая XYZ D50 в цветовое пространство камеры, тогда $CM^{-1}$ будет матрицей переводящей цветовое пространство матрицы в XYZ D50. Пусть $XTS$:

\begin{equation}
	\label{XTS}
	XTS = \begin{bmatrix}
			3.1338561 & -1.6168667 & -0.4906146\\
			-0.9787684 & 1.9161415 & 0.0334540 \\ 
			0.0719453 & -0.2289914 & 1.4052427
		\end{bmatrix}
\end{equation}

матрица преобразующая XYZ D50 в sRGB. Тогда для преобразования изображения из цветового пространства камеры в sRGB необходимо произвести умножение:

\begin{equation}
	\label{tosrgb}
	T_{sRGB} = XTS * CM * T_{CC}
\end{equation}

где $T_{CC}$ --- кортеж с значениями цвета в пространстве камеры.

$T_{sRGB}$ находится в цветовом пространстве sRGB, но яркость все еще закодирована линейно, для правильного отображения необходимо применить гамма-коррекцию:

\begin{equation}
	\label{gamma}
	\gamma(u) = 
	\begin{cases}
		12.92u, & u \leq 0.04045 \\
		1.055u^{1/2.4} - 0.055
	\end{cases}
\end{equation}

где $u$ --- одна из компонент цвета.

\section{Настройка изображения.}

\subsection{Яркость.}

Преобразование яркости определяется как:
\begin{equation}
	\label{brightness}
	br(a) = a + N
\end{equation}
где $a$ --- значение яркости пикселя, а $N$ --- желаемое увеличение в яркости. \cite{brk}

\subsection{Контрастность.}

Преобразование контрастности определяется как:
\begin{equation}
	\label{contrast}
	br(a) = a * N
\end{equation}
где $a$ --- значение яркости пикселя, а $N$ --- желаемое увеличение в яркости (чтобы, например, увеличить контрастность на 50\% необходимо умножить на 1.5). \cite{brk}

\subsection{Насыщенность.}

Для изменения насыщенности изображения в цветовой модели RGB можно воспользоваться умножением матриц.

Пусть $F(x,y)$ --- вектор:
\begin{equation}
	\label{F}
	F(x,y) = [f_R, f_G, f_B, 1]^T
\end{equation}
где $f_R, f_G, f_B$ --- значения цвета в точке $x,y$.

Тогда $G(x,y)$ --- вектор содержащий значения цвета с иной насыщенностью:
\begin{equation}
	\label{G}
	G(x,y) = [g_R, g_G, g_B, g_w]^T
\end{equation}
где $g_R, g_G, g_B$ --- новые значения цвета в точке $x,y$, а $g_w$ не используется.

Высчитать $G(x,y)$ можно по формуле \ref{sat}:
\begin{equation}
	\label{sat}
	G(x,y) = T * F(x,y)
\end{equation}

где $T$ --- матрица преобразования:
\begin{equation}
	\label{tsat}
	T_{sat}(s) = \begin{bmatrix}
					\alpha + s & \beta & \gamma & 0\\
				    \alpha & \beta + s & \gamma & 0 \\ 
					\alpha & \beta & \gamma + s & 0 \\ 
					0 & 0 & 0 & 1 \\ 
				\end{bmatrix}
\end{equation}
где $\alpha = 0.3086(1 - s)$, $\beta = 0.6094(1 - s)$ и $\gamma = 0.0820(1 - s)$.

Значения $s < 1$ приводят к уменьшению насыщенности, значения $> 1$ --- к увеличению. \cite{sat}

\section*{Вывод}
В данном разделе был проведен обзор необходимых для реализации алгоритмов.