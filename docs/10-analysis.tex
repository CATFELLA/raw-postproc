\chapter{Аналитическая часть}

\section{Алгоритмы демозаики.}

Одной из главных задач обработки RAW видео является устранение эффектов фильтра Байера.

\img{60mm}{bayer_num}{Пронумерованный фильтр Байера.} 

\subsection{Билинейная интерполяция}
Билинейная интерполяция использует среднее значение двух или четырех соседних пикселей соответствующего цвета, например: значения синего цвета для пикселей 8 и 13 находятся как:
\begin{equation}
	B8 = \frac{B7 + B9}{2},
	\quad	
	B13 = \frac{B7 + B9 + B17 + B19}{2}
\end{equation}

Данный алгоритм считается одним самых быстрых и часто используется в интерполяции видео в реальном времени.

\subsection{Smooth hue transition}

Алгоритм производит два прохода, сначала применяется алгоритм билинейной интерполяции для восстановления зеленого канала. Затем, второй проход использует отношение между зеленым и красным/синим в пикселе для восстановления оставшихся каналов. Например, значения синего цвета для пикселя 13 расчитывается как:
\begin{equation}	
	B13 = \frac{G13}{4}(\frac{B7}{G7} + \frac{B9}{G9} + \frac{B17}{G17} + \frac{B19}{G19})
\end{equation}

Этот алгоритм использует тот факт, что цвет между пикселями меняется плавно и резкие переходы приведут к появлению визуальных артефактов.

\section*{Вывод}
Были описаны 3 алгоритма сортировки: пузырьком, вставками и расческой.