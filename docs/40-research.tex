\chapter{Исследовательская часть}

В данном разделе будут приведены примеры работы разработанного программного обеспечения, а также будет поставлен эксперимент, в котором будут сравнены скорость и качество алгоритмов.

\section{Результаты работы программного обеспечения}

На рисунках \ref{img:bil}, \ref{img:mhc}, \ref{img:sh} и \ref{img:shmhc} представлены результаты работы разных алгоритмов демозаики, на рисунке \ref{img:control_panel} представлено окно панели управления.

\img{117mm}{bil}{Билинейная интерполяция.}
\img{117mm}{mhc}{Malvar-He-Cutler.}
\img{117mm}{sh}{Smooth hue transition.}
\img{117mm}{shmhc}{Smooth hue transition + Malvar-He-Cutler.}

\clearpage

\section{Постановка эксперимента}

\subsection{Цель эксперимента}

Цель эксперимента - сравнение графических характеристик комбинированного алгоритма демозаики с его прообразами.

\subsection{Замеры скорости обработки кадра}

\begin{table}[h]
	\begin{center}
		\caption{Скорость обработки кадра (в тиках).}
		\label{tbl:small}
		\resizebox{16cm}{!}{
			\begin{tabular}{|c|c|c|c|c|}
				\hline
				\bfseries Разрешение & \bfseries Билинейная интерполяция & \bfseries Malvar-He-Cutler & \bfseries Smooth Hue Transition  & \bfseries Smooth Hue Transition + Malvar-He-Cutler \\
				\hline
				640x360 & 7955367 & 13021776 & 12675546 & 11940170 \\ \hline
				1736*976 & 59383502 & 79846533 & 64410020 & 83702792 \\ \hline
				1600x1160 & 65717045 & 87026943 & 73147675 & 89480050 \\ \hline
				5280x3528 & 940273338 & 1156715212 & 1009131767 & 1193178840 \\ \hline
		\end{tabular}}
	\end{center}
\end{table}


\begin{figure}[h]
	\centering
	\begin{tikzpicture}
		\begin{axis}[
			axis lines=left,
			xlabel=Размер изображения,
			ylabel={Время, нс},
			legend pos=north west,
			ymajorgrids=true
			]
			\addplot table {
				230400 7955367 
				1694336 59383502
				1856000 65717045
				18627840 940273338
			};
			\addplot table {
				230400 13021776 
				1694336 79846533
				1856000 87026943
				18627840 1156715212
			};
			\addplot table {
				230400 12675546 
				1694336 64410020
				1856000 73147675
				18627840 1009131767
			};
			\addplot table {
				230400 11940170 
				1694336 83702792
				1856000 89480050
				18627840 1193178840
			};
			\legend{B, MHC, SHT, SHT + MHC}
		\end{axis}
	\end{tikzpicture}
	\captionsetup{justification=centering}
	\caption{Зависимость времени работы алгоритма от размера изображения.}
	\label{plt:time_img}

\end{figure}

\subsection{Обработка артефактов.}

Основной задачей алгоритмов демозаики является уменьшения воздействия фильтра Байера.

\img{100mm}{bil_close}{Билинейная интерполяция.}
\img{100mm}{mhc_close}{Malvar-He-Cutle.}
\img{100mm}{sh_close}{Smooth hue transition.}
\img{100mm}{shmhc_close}{Smooth hue transition + Malvar-He-Cutler.}
\clearpage
На изображениях \ref{img:bil_close}, \ref{img:mhc_close}, \ref{img:sh_close} и \ref{img:shmhc_close} показано восстановление жесткого края.
\clearpage

\img{100mm}{bil_cl}{Билинейная интерполяция.}
\img{100mm}{mhc_cl}{Malvar-He-Cutle.}
\img{100mm}{sh_cl}{Smooth hue transition.}
\img{100mm}{shmhc_cl}{Smooth hue transition + Malvar-He-Cutler.}
\clearpage

На изображениях показана резкость каждого алгоритма, самым резким является алгоритм Malvar-He-Cutler (\ref{img:mhc_cl}), затем Smooth Hue Transition + Malvar-He-Cutler (\ref{img:shmhc_cl}), Smooth Hue Transition (\ref{img:sh_cl}) и самым нерезким является билинейная интерполяция, изображение \ref{img:bil_cl}.

Так как разрешение зеленого канала в два раза больше разрешений красного и синего каналов, более точное восстановление зеленого канала ведет к большей резкости изображения.

\clearpage
\img{100mm}{bil_green}{Билинейная интерполяция.}
\img{100mm}{mhc_green}{Malvar-He-Cutle.}
\img{100mm}{sh_green}{Smooth hue transition.}
\img{100mm}{shmhc_green}{Smooth hue transition + Malvar-He-Cutler.}
\clearpage

На изображении \ref{img:mhc_green} алгоритм Malvar-He-Cutler восстановил край без зеленых артефактов. Остальные алгоритмы интерполировали край с зелеными или, в случае билинейной интерполяции, цветными артефактами. Однако артефакты алгоритма Smooth Hue Transition + Malvar-He-Cutler на изображении \ref{img:shmhc_green} немного слабее чем у алгоритма Smooth Hue Transition на изображении \ref{img:sh_green}.

\section*{Вывод}

В данном разделе были рассмотрены примеры работы программного обеспечения и были сравнены результаты работы разных алгоритмов. В результате сравнения были получены следующие результаты:
\begin{itemize}
	\item все алгоритмы корректно воспроизводят цвет;
	\item алгоритмы сопоставлены в таблице \ref{tbl:algs}
\end{itemize}

\begin{table}[h]
	\begin{center}
		\caption{Сравнение алгоритмов.}
		\label{tbl:algs}
		\resizebox{16cm}{!}{
			\begin{tabular}{|c|c|c|c|c|}
				\hline
				\bfseries Параметр & \bfseries Билинейная интерполяция & \bfseries Malvar-He-Cutler & \bfseries Smooth Hue Transition  & \bfseries Smooth Hue Transition + Malvar-He-Cutler \\
				\hline
				Скорость (малое разрешение) & 1 & 4 & 3 & 2 \\ \hline
				Скорость (большое разрешение) & 1 & 3 & 2 & 4 \\ \hline
				Подавление артефактов & 4 & 1 & 3 & 2 \\ \hline
				Резкость & 4 & 1 & 3 & 2 \\ \hline
		\end{tabular}}
	\end{center}
\end{table}

