\chapter*{Введение}
\addcontentsline{toc}{chapter}{Введение}

RAW видео --- это видео содержащее необработанную информацию об изображении с сенсора камеры.

Главный элемент цифровых камер --- сенсор, при попадании света на сенсор на нем накапливается заряд. Из этих зарядов формируется изображение.

Однако без дополнительных средств любой свет воспринимается сенсором одинаково, и на выходе получается черно-белое изображение. Наиболее распространенными способами записи цветного изображения в одну экспозицию являются: фильтр Байера, над одной матрицей или разделение изображения на три цвета, красный, зеленый и синий, и обработка каждого из них отдельной матрице. \cite{color}

Несмотря на то что метод разделения на три матрицы дает наиболее качественный результат, в большинстве камер среднего ценового сегмента установлена одна матрица с фильтром Байера.

Фильтр Байера состоит из 25\% красных элементов, 25\% синих и 50\% зеленых элементов, как показано на рисунке \ref{img:bayer}.

\img{60mm}{bayer}{Фильтр Байера.} 

Изображение с такого фильтра дает возможность создания цветного изображения, однако без обработки оно не будет таковым. Поэтому необходимо произвести процесс демозаики, который приведет изображение к корректному виду. 

После этого можно приступать к остальным настройкам изображения, таким как: преобразование цвета, настройка баланса белого, тональных кривых, контрастности, насыщенности и так далее.

Таким образом, цель данной работы --- реализовать ПО позволяющее просматривать, обрабатывать и сохранять RAW видео.

Чтобы достигнуть поставленной цели, требуется решить следующие задачи:
\begin{itemize}
	\item реализовать открытие и отображение RAW файлов;
	\item реализовать инструменты для обработки видео;
	\item реализовать возможность сохранения модифицированного видео;
	\item реализовать пользовательский интерфейс.
\end{itemize}
